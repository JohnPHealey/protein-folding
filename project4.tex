\documentclass[12pt]{article} 
\usepackage{graphicx} 
\usepackage{amsmath}
\begin{document} 
\linespread{1.3} 
\setlength\parindent{0pt}
\title{P3800 Project 4: The Protein-Folding Problem and Levinthal's Paradox} 
\date{Written April 2018} 
\author{John Healey}
\maketitle


\section{Introduction}
The Protein-Folding Problem is an inter-disciplinary topic, involving Physics and Biology, so I will start by giving an 
overview of the biological concepts needed to understand the problem. A protein is a molecule that plays a particular 
role in the body. Proteins are made up of simple organic compunds called amino acids, connected end to end to form 
a chain. The types of amino acids that make up the protein, and the way in which they are arranged, determine the 
protein's ability to perform its biological function. This is not only based on the order of the acids in the chain, 
which determines its \emph{primary structure}, but also on the shape of the chain itself, i.e., how it is folded. 
The way in which the chain is folded is known as its \emph{tertiary structure}.

In order for a protein to perform its biological function, then, it must be folded in exactly the right way, and if 
it becomes unfolded, it must return to its correct tertiary structure. However, for a protein consisting of 300 amino 
acids, and assuming 4 possible orientation angles between connected acids, there are 4^{N} \approx 10^180$ possible 
tertiary structures arising from a single primary structure. Despite this inconceivably large amount of possible 
structures, not only is a protein able to consistently go from an unfolded state to its correct tertiary structure, 
it is able to do so in seconds. Even if the protein spent as little as $10^-13$s in each intermediate state, the folding 
process would still take longer than the age of the universe. This is known as Levinthal's Paradox. How the protein 
knows which tertiary is correct is the Protein-Folding Problem.


\section{Method of Investigation}
To try to understand how protein folding might work, I 

\begin{equation}
E = -J \sum_{i,j}^{NN} S_i S_j 
\end{equation}

\begin{equation}
S =  \sum_{i=1}^{N} S_i.
\end{equation}


By taking thermal averages over Monte Carlo steps for these quantities, we were able to examine more interesting aspects of 
the system, including its magnetization M, heat capacity $C_v$, and magnetic susceptibility $\chi$. These quantities were 
measured using the following equations:


\begin{equation}
M = \frac{1}{N} \langle |M| \rangle
\end{equation}

\begin{equation}
C_v = \frac{1}{N} \frac{1}{k_B T^2} \Big[ \langle E^2 \rangle - \langle E \rangle ^2 \Big]
\end{equation}

\begin{equation}
\chi = \frac{1}{N} \frac{1}{k_B T} \Big[ \langle S^2 \rangle - \langle |S| \rangle ^2 \Big].
\end{equation}


\section{Simulation}

The simulation measured the aforementioned quantities at a range of temperatures around the critical temperature $T_c \approx 2.269$, 
at which the magnetization of the system goes from 0 to 1 (approaching from the higher temperature), and the heat capacity 
and magnetic susceptibility diverge (for an infinite lattice). By starting the system at a higher temperature and cooling 
it down, we expect that the spins will tend toward a more aligned state, since the energy will tend to decrease. So, the 
magnetization of the lattice should increase as the temperature decreases. With this in mind, we start with an initially random 
configuration of the lattice at the relatively high starting temperature. In order to let the system equilibrate 
without effecting thermal averages, we discard the first $10 \%$ of the runs. 


\section{Coding Issues}

The main issue faced in this project was getting the program to run in a reasonable amount of time for large lattice 
sizes L and Monte Carlo Sweeps MCS. My simulation was written in Python, an interpreted language, which caused it to run for 
a very long time for lattice sizes of 100, or MCS of 100,000. The program was able to run within a day with $L = 100, 
M = 10,000$, but this was the largest viable MCS I could use with L = 100. I was able to get a speed-up of approximately 
$20 \%$ by minimizing the amount of times I made calls to random number generators, and increasing my usage of the 
Python scientific computing library, NumPy. I also examined the optimizing compiler Numba, which shows definite potential 
for increasing the speed of my program, but has yet to have the desired effect.


\section{Results}

I ran my simulation using a variety of lattice sizes and Monte Carlo Sweeps. The results for the various quantities of 
interest are shown below. As expected, the larger lattice sizes had steeper transitions from $M = 0$ to $M = 1$, and 
higher peaks in the divergence of $C_v$ and $\chi$. The peaks were also more centered on $T_c$ for the larger lattices, 
making them more accurate for predictions of this quantity. This is because they more closely approximated an infinite lattice, 
which is the model used in the analytical predictions. I also noticed that the higher MCS runs produced much smoother 
plots that better fit the expected results. This makes sense, since the quantities measured are taken as thermal averages 
over MCS, so the more MCS we took, the smoother these averages would be. Also shown below is an illustration of how the spins 
aligned as the simulation progressed. As can be seen, at higher temperatures, there were approximately as many spins in the 
up-state as those in the down state. Near the critical temperature, where the magnetization of the system transfers from zero 
to one, most of the spins are in the up-state. Finally, as the temperature is lowered past the critical temperature, most of 
the spins are in the down-state. This increase in alignment with decreasing temperature agreed with the expected behaviour 
of such a system.


$T = 3.15$

$$
\begin{matrix}
-1 & -1 & -1 & +1 & -1 & -1 & +1 & +1 & +1 & +1 \\
-1 & -1 & -1 & +1 & +1 & +1 & +1 & +1 & +1 & +1 \\
-1 & -1 & -1 & -1 & -1 & +1 & +1 & -1 & +1 & -1 \\
-1 & -1 & -1 & +1 & +1 & +1 & +1 & +1 & +1 & +1 \\
-1 & -1 & -1 & +1 & -1 & -1 & +1 & -1 & -1 & -1 \\
-1 & -1 & -1 & -1 & -1 & -1 & +1 & +1 & +1 & +1 \\
-1 & -1 & +1 & +1 & -1 & -1 & +1 & +1 & +1 & -1 \\
-1 & -1 & +1 & -1 & -1 & +1 & +1 & +1 & -1 & -1 \\
+1 & -1 & -1 & -1 & +1 & +1 & +1 & +1 & +1 & +1 \\
-1 & -1 & -1 & +1 & +1 & +1 & +1 & +1 & +1 & +1
\end{matrix}
$$

\newpage

$T = 2.25$

$$
\begin{matrix}
-1 & +1 & +1 & +1 & -1 & +1 & +1 & +1 & +1 & +1 \\
+1 & +1 & +1 & +1 & +1 & +1 & +1 & +1 & +1 & +1 \\
+1 & +1 & +1 & +1 & +1 & +1 & +1 & +1 & +1 & +1 \\
+1 & +1 & +1 & +1 & -1 & +1 & +1 & -1 & -1 & +1 \\
+1 & +1 & +1 & +1 & +1 & +1 & +1 & +1 & -1 & +1 \\
+1 & +1 & +1 & +1 & +1 & +1 & +1 & -1 & +1 & +1 \\
+1 & +1 & +1 & +1 & +1 & +1 & +1 & +1 & +1 & +1 \\
+1 & +1 & +1 & +1 & +1 & +1 & +1 & +1 & +1 & +1 \\
+1 & +1 & +1 & +1 & +1 & +1 & +1 & +1 & +1 & +1 \\
+1 & +1 & +1 & +1 & -1 & +1 & +1 & +1 & +1 & +1 \\
\end{matrix}
$$


$T = 1.15$
$$
\begin{matrix}
-1 & -1 & -1 & -1 & -1 & -1 & -1 & -1 & -1 & -1 \\
-1 & -1 & -1 & -1 & -1 & -1 & -1 & -1 & -1 & -1 \\
-1 & -1 & -1 & -1 & -1 & -1 & -1 & -1 & -1 & -1 \\
-1 & -1 & -1 & -1 & -1 & -1 & -1 & -1 & -1 & -1 \\
-1 & -1 & -1 & -1 & -1 & -1 & -1 & -1 & -1 & -1 \\
-1 & -1 & -1 & -1 & -1 & -1 & -1 & -1 & -1 & -1 \\
-1 & -1 & -1 & -1 & -1 & -1 & -1 & -1 & -1 & -1 \\
-1 & -1 & -1 & -1 & -1 & -1 & -1 & -1 & -1 & -1 \\
-1 & -1 & -1 & -1 & -1 & -1 & -1 & -1 & -1 & -1 \\
-1 & -1 & -1 & -1 & -1 & -1 & -1 & -1 & -1 & -1
\end{matrix}
$$


\begin{figure}[h]
\includegraphics[width=\textwidth]{ising_plotMTall.png} 
\label{magnetization_results}
\end{figure}

\begin{figure}[h]
\includegraphics[width=\textwidth]{ising_plot100000M.PNG} 
\caption{ Magnetization as a function of Temperature for MCS 100, 1,000, 10,000, 100,000 }
\label{magnetization_results}
\end{figure}

\begin{figure}[h]
\includegraphics[width=\textwidth]{ising_plot100C.PNG} 
\label{heatcapacity_results}
\end{figure}

\begin{figure}[h]
\includegraphics[width=\textwidth]{ising_plotCTall.png} 
\caption{ Heat capacity as a function of Temperature for MCS 100, 1,000, 10,000, 100,000 }
\label{heatcapacity_results}
\end{figure}

\begin{figure}[h]
\includegraphics[width=\textwidth]{ising_plotXTall.png} 
\label{magnetization_results}
\end{figure}

\begin{figure}[h]
\includegraphics[width=\textwidth]{ising_plot100000X.PNG} 
\caption{ Magnetic Susceptibility as a function of Temperature for MCS 100, 1,000, 10,000, 100,000 }
\label{magnetization_results}
\end{figure}




\end{document} 
